\chapter*{Introduction}
\addcontentsline{toc}{chapter}{Introduction}

Internet has become an unavoidable component of everyday life of billions of people. Among other things, it serves as a
source of knowledge which is available literally at our fingertips. However, it is not possible for a single person to
filter or find relevant information from such a huge dataset. As a result, many search engines emerged which try to
provide this functionality. Sometimes, users do not know for what they are exactly looking or the search engine can
predict for what the user could be looking based on some criteria. These criteria consist mainly of what the user
already typed into the system or the user's search history. Nevertheless; many others may be integrated as well, e.g.
the system can better predict search based on some personal information about the user (age, gender, etc.) or
searches of other users.


Suggester, or autocomplete, functionality tries to achieve this deficiency by giving the user the possibility to choose
from a handful of choices it deemed the most attractive for the user. Therefore, suggester implementations were added to
many search engines (e.g. Google\footnote{\url{http://google.com}} – it should be noted that OpenGrok is a specialized
search engine and should not be compared with Google directly).
However, OpenGrok\footnote{\url{http://opengrok.org}}
search engine does not contain any suggester implementation.

\section{Target of the thesis}

The thesis aims to fulfill the following targets:
\begin{itemize}
    \item Integrate suggester functionality into OpenGrok.
    \item Suggestions provided by the suggester should be meaningful and helpful.
    \item The suggester should provide support for rich configuration.
    \item Evaluate the impact of the suggester on the system running the OpenGrok instance.
\end{itemize}

\section{Structure}

Chapter \ref{chap:opengrok} \textbf{OpenGrok} provides a quick overview of OpenGrok project. What tools it uses and what
makes the OpenGrok project extraordinary.

Chapter \ref{chap:analysis} \textbf{Analysis} discusses problems that arose during the implementation process and how
they were resolved.

Chapter \ref{chap:user} \textbf{User Documentation} contains pictures of the added functionality and explains how it
should be used.

Chapter \ref{chap:program} \textbf{Program Documentation} describes implementation-specific properties of the work.
In other words, how exactly was the new content integrated into the OpenGrok project.

Chapter \ref{chap:impact} \textbf{Impact on the OpenGrok Project} provides detailed description, comparison and
evaluation of the functionality that affected the OpenGrok project.
