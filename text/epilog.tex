\chapter*{Conclusion}
\label{chap:epilog}

\addcontentsline{toc}{chapter}{Conclusion}

The target of the thesis was to implement an autocomplete functionality for the OpenGrok search engine. The implemented
solution enabled this functionality for all the basic query types supported by OpenGrok. These query types were prefix queries,
wildcard queries, regexp queries, range queries, fuzzy queries, and phrase queries. The implemented solution
provides different suggestions according to the currently selected projects.
Based on the data size and number of projects
that are searched, the suggestions might take a long time to compute. Therefore, if the suggester is not able to return
the results in a specified time, only partial results are returned.

The suggestions also provide correct results for multiple Lucene fields. That means, multiple data structures
which store different information need to be maintained for every project in the OpenGrok instance.

To improve the suggestion quality, the most popular completion was used. The suggester remembers the search counts
for every term and promotes the terms with the highest search counts. If no search data are available, the suggester
uses the underlying statistics of the document set. It is also possible to initialize or update this data by using the API
provided to administrators.

The solution provides a possibility to configure many aspects of the suggester to truly satisfy the users' needs.
Among others, it is possible to specify minimum characters needed for suggestions to appear or to completely disable
the suggestions.

There is no need to explicitly enable or configure the already existing OpenGrok instances. Only an upgrade to the
version with the implemented suggester functionality is needed.

The existing solution can be improved and extended in many ways. First of all, optimizations for better suggestions
speed should be implemented to provide better user experience. One of the optimizations that might be implemented is to use
disk access for phrase queries with only a few documents which might increase their performance by an order of magnitude.

Secondly, suggestions might be improved to take into account even more factors and not just popularity and term or document frequency.
Personalized suggestions for users is a very strong candidate for an extension. It is already implemented by Google and
other search engines. Furthermore, error toleration is a very intriguing possibility for improvement. To err is human and
to provide suggestions for a query with a small mistake might be an invaluable feature not only for the users on the mobile devices.
Moreover, it might make sense to also make the suggester time-sensitive. In other words, the suggester could provide
different suggestions based on the actual time or day.

Last but not least, suggestions could be more sophisticated if the suggester were to have a direct support for complex queries.
For instance, it could boost different terms based on the already input data or it could recognize a pattern and suggest
the rest of the complex query.
