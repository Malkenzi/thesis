\chapter{Impact on the OpenGrok project}
\label{chap:impact}

\textit{Change impact analysis}\footnote{\url{https://en.wikipedia.org/wiki/Change\_impact\_analysis}} is a field of
study which task is to determine what impact the changes may have on the system. Sometimes, it is very hard to analyze
the impact of a change. Even more so in a large projects with a huge codebase. Analysis of the main decisions made is
described in detail in the chapter \ref{chap:analysis}. The main aspects of the added changes that can be discussed are:
\begin{itemize}
    \item Impact on the search results. Discussed in more detail in \ref{search_res_impact}.
    \item Impact on the hardware requirements. Discussed in more detail in \ref{hw_req_impact}.
\end{itemize}

\section{Impact on the Search Results}
\label{search_res_impact}
It is hard to compare search results; therefore, search engines as well. Which search engine performs better, the one
that returns only a few relevant documents but omitted many in the process or the one that returns all relevant
documents but includes some that are not?
In information retrieval field there are measurements which take this into account, most notable are \textbf{precision}
and \textbf{recall}\footnote{\url{https://en.wikipedia.org/wiki/Precision\_and\_recall}}.

\textbf{Precision} is the fraction of retrieved documents that are relevant to the query. Equation can be seen on
\ref{precision_eq}.

\begin{equation}
\label{precision_eq}
precision = \frac{\vert \{relevant\ documents\} \cap \{ retrieved\ documents \} \vert}{\vert \{ retrieved\ documents \} \vert}
\end{equation}

\textbf{Recall} is the fraction of the relevant documents that are successfully retrieved. Equation can be seen on
\ref{recall_eq}.

\begin{equation}
\label{recall_eq}
recall = \frac{\vert \{relevant\ documents\} \cap \{ retrieved\ documents \} \vert}{\vert \{ relevant\ documents \} \vert}
\end{equation}

However, suggester does not impact these measurements directly. The same query returns the same results with or without
the suggester. Nevertheless, suggester affects them indirectly:
\begin{itemize}
    \item There are less queries which yield no results – if the user types a few characters and there are no suggestions then he
    can immediately see that there are no terms with that prefix. Therefore, the result will be empty.
    \item Probably negative impact on precision because of the scoring described in \ref{prefix_scoring}. The terms which
    are in more documents have greater scores. Thus, these terms are promoted and are more likely to be chosen by the user.
    Therefore, the size of retrieved documents set will be greater. However, the initial scoring might be mitigated
    by taking into account previous users' searches as described in \ref{previous_searches}.
\end{itemize}

It is not easy to achieve feasible values for both precision and recall. In many information retrieval systems when one is
being increased the other one decreases. This is most notable in Boolean Information Retrieval Model
\footnote{\url{https://en.wikipedia.org/wiki/Standard\_Boolean\_model}}.

\section{Impact on the hardware requirements}
\label{hw_req_impact}
Since suggester is mostly independent from the existing solution it only increases hardware consumption.
The most affected resources are:
\begin{itemize}
    \item \textbf{CPU} – in simple situations where only a prefix is typed the CPU load is not that high because it
    is a lookup in % bullshit
    \item \textbf{Memory} –
    \item \textbf{Disk} –
\end{itemize}


